\documentclass[12pt,a4paper,sans]{moderncv}

\moderncvtheme[black]{classic}

\usepackage[utf8]{inputenc}

\usepackage[scale=0.8]{geometry}
\recomputelengths

\firstname{Md. Ashfaqur}
\familyname{Rahaman}
\title{Curriculum Vitae}
\address{991 W. Blaine St.}{92507 Riverside}
\mobile{+1-951-222-9581}
\email{sajib.finix@gmail.com}
\homepage{github.com/aagontuk}

\begin{document}

\maketitle

\section{Research Interests}

\cvlistitem{Operating Systems and System Programming}
\cvlistitem{Computer Architecture}
\cvlistitem{Compilers}
\cvlistitem{Security}
\cvlistitem{Linux Kernel}
\cvlistitem{Embedded Systems}

\section{Experience}
\subsection{Research}

\cventry{Dec. 2018-Aug. 2019}{Research Assistant}{Climate Modeling and Simulation Lab}{IWFM, BUET}{}{I worked as a system developer in Flash Flood Early Warning System (FFEWS) project}

\subsection{Professional}

\cventry{Apr. 2018-Apr. 2019}{Lead Software Engineer}{NextGen Digitech}{Dhaka}{}{Worked on a computer application for designing wind turbines. I have contributed in the core software architecture and designed the GUI}

\cventry{Oct. 2017-Mar. 2018}{Firmware Engineer}{2RA Technology Limited}{Dhaka}{}{I have worked on different projects based on Raspberry Pi and AVR Microcontrollers. Here are few of them:\\*\begin{description}\item{\emph{Energy Monitoring System}} A system for monitoring a textile industry's power generators from the web\item{\emph{UHF Attendance System}} Automated attendance system for schools using UHF RFID\item{\emph{Battery Voltage Monitoring System}} A battery voltage and temperature monitoring system for an industry's backup power system in different sites accross the country and centralize the data\item{\emph{Temperature Monitoring System}} Designed web based temperature monitoring system for a company's parking lot.\\\end{description}}

\pagebreak

\subsection{Competitions}

\cventry{2018}{Google Kickstart Coding Competition}{Google}{}{}{Participated in the qualification round and solved all the problems.}

\cventry{2016}{RoboSoccer Competition}{Engineering Student Association Bangladesh (ESAB)}{}{}{We have participated in the competition and got honorable mention}

\cventry{2015}{Model Ship Propulsion Competition}{BUET}{}{}{My team secured third position. I have worked in the programming and hardware interfacing part}

\cventry{2015}{Android App Contest}{EATL-Prothomalo}{}{}{We developed an Android application for checking OMR sheets by using image processing algorithms. Our app was in the top 100 list.}

\section{Training and Courses}

\cvlistitem{Operating System and System Programming CS162, UC Berkeley}
\cvlistitem{Mathematics for Computer Science 6.042J, MIT OCW}
\cvlistitem{Introduction to Algorithms 6.006, MIT OCW}
\cvlistitem{Practical Programming in C 6.087, MIT OCW}
\cvlistitem{Advanced C++ Programming for Ship Structure, BUET}
\cvlistitem{Fortran Programming, BUET}


\section{Skills}

\cvitem{Programming Languages}{C, C++, Python, JAVA, Bash, Assembly(AVR, X86)}
\cvitem{Frameworks}{Qt, Android}
\cvitem{Embedded Systems}{AVR Microcontrollers, Raspberry Pi, Arduino}
\cvitem{GNU/Linux}{6 years of experience in the GNU/Linux environment}
\cvitem{VCS}{Git and Github}
\cvitem{Text Editing}{Vim, \LaTeX}
\cvitem{GRE}{313, Quantitative - 162, Verbal - 151, AWA - 3.0}
\cvitem{TOEFL}{101, Reading - 28, Listening - 27, Speaking - 23, Writing - 23}

\section{Education}

\cventry{2012-2019}{B.Sc in Naval Architecture and Marine Engineering}{Bangladesh University of Engineering and Technology (BUET)}{\textit{CGPA -- 2.69 on 4.00}}{}{}
\cventry{2011}{Higher Secondary Certificate Exam}{Dhaka College}{Dhaka Board}{\textit{GPA -- 5.00 on 5.00}}{Science Devision}
\cventry{2009}{Secondary School Certificate Exam}{A. K. High School and College}{Dhaka Board}{\textit{GPA -- 5.00 on 5.00}}{Science Division}

\section{Bachelor Thesis}

\cvline{Title}{\emph{Power Efficient Remotely Operated Underwater Vehicle Using Buoyancy Chambers}}
\cvline{Supervisor}{Professor Dr. Md. Mashud Karim}
\cvline{Description}{In this thesis, a power efficient remotely operated underwater vehicle (ROV) has been built using buoyancy chambers.Conventionally, thruster motors are used for the propulsion of the ROV. In this system to keep the vehicle at rest in a fixed position under water, thruster motors are used and battery power is consumed continuously. This power loss can be removed by adding simple buoyancy mechanism. A control algorithm has been developed to control all the valves and pump. Also a computer program has been developed to communicate with the on board controller of the ROV.} 


\end{document}

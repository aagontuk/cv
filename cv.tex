\documentclass[11pt,a4paper,sans]{moderncv}

\moderncvtheme[black]{classic}

\usepackage[utf8]{inputenc}

\usepackage[scale=0.9]{geometry}
\recomputelengths

\firstname{Md Ashfaqur}
\familyname{Rahaman}
\address{50 Central Campus Dr \#3163}{Salt Lake City, UT 84112}
\mobile{+1-951-222-9581}
\email{ashfaq@cs.utah.edu}
\homepage{aagontuk.github.io}

\begin{document}

\maketitle

\section{Research Interests}

\cvitem{}{Kernel bypass, disaggregated memory, hardware offloading and systems for machine learning}

\section{Education}

\cventry{Aug. 2021}{Ph.D. in Computer Science}{University of Utah}{Salt Lake City}{Utah, USA}{\emph{Advisor: Ryan Stutsman}}
\cventry{2012-2019}{B.Sc. in Naval Architecture and Marine Engineering}{Bangladesh University of Engineering and Technology (BUET)}{Dhaka}{Bangladesh}{}

\section{Experience}
\subsection{Research}

\cventry{\mbox{Summer 2024}}{Research Intern}{Hewlett Packard Labs}{Milpitas, California}{}{Optimizing the communication infrastructure to support low-latency high-throughput LLM inference}
\cventry{2021-Present}{Graduate Research Assistant}{Utah Scalable Computer Systems Lab}{University of Utah, Utah}{}{
\begin{itemize}
  \item{A new efficient, secure, and scalable framework for remote memory access and function offloading exploiting the programmability and offloading capability of smartNICs}
	\item{A new software architecture for building services that centers around coherent accelerators and rack-scale shared memory}
  %\item{Accelerating the network read path of in-memory key-value stores and making Linux page cache policy configurable from userspace using eBPF.}
\end{itemize}}

\cventry{2019-2021}{Research Assistant}{Prof. Baris Kasikci's Lab}{University of Michigan, Ann Arbor}{}{\emph{Mentor: Tanvir Ahmed Khan}\\Load-time code layout optimization of large application binaries in warehouse scale computers}

\cventry{2018-2019}{Research Assistant}{Climate Modeling and Simulation Lab}{IWFM, BUET}{}{\emph{Advisor: A.K.M. Saiful Islam}\\I worked as a system developer in real-time Flash Flood Early Warning System (FFEWS) project
}

\smallskip
\subsection{Professional}

\cventry{2018-2019}{Software Engineer}{NextGen DigiTech}{Dhaka}{}{I worked on NextGen Tower, a desktop application for designing wind turbines. I contributed in the core software architecture and developed the GUI.}

\cventry{2017-2018}{Firmware Engineer}{2RA Technology Limited}{Dhaka}{}{I worked on various embedded systems projects based on Raspberry Pi and AVR Microcontrollers.}

\section{Selected Research Projects}

\cventry{2021-Present}{NIC Accelerated Active Messaging}{}{}{}{
RDMA is gaining popularity in datacenters for high-throughput
and low-latency network communication for building dis-aggregated systems.
However, there are many issues that are holding RDMA back from being widely deployed.
Current RDMA verbs are limited for diverse workloads, they are difficult to program,
and multiple round trips are required to do complicated memory operations e.g. walking a hash table.
We are working on creating a new network abstraction to make remote memory access more efficient, secure, and scalable
exploiting the programmability and offloading feature of smartNICs.
In our system, NIC offloads can be written in a high level language ensuring easy programmability,
then this will be converted into verifiable bytecode tailored to a specific workload and run on the NIC or host based on the dynamic load.
Program transformation, dynamic decisions, all these will be transparent to the developer.}

\cventry{2023-Present}{Software Architectures for Large-Scale Coherent Shared Memory}{}{}{}{
Emerging standards for cache coherent accelerators (e.g. CXL) will soon transform
how memory-intensive large-scale systems are developed.
Cache coherent accelerators are programmable (via FPGAs),
and they can interpose on CPU memory accesses at cache line granularitiy.
Low-overhead, granular access tracking with these coherent accelerators
enable efficient memory disaggregation. But disaggregation alone will
not fundamentally change application architecture. In this work we
are designing a new software architecture for building large scale services
that centers around coherent accelerators and rack-scale shared memory.}

\bigskip
\bigskip
\bigskip
\section{Selected Courses}

\cvline{Fall 2024}{Advanced Compilers, University of Utah}
\cvline{Spring 2024}{Advanced Computer Architecture, University of Utah}
\cvline{Fall 2023}{Advanced Networking, University of Utah}
\cvline{Spring 2022}{Software Security, University of Utah}
\cvline{Fall 2021}{Advanced Operating Systems, University Of Utah}

\section{Teaching Assistantship}

\cvline{Fall 2022}{Distributed Systems, University Of Utah}
\cvline{Spring 2022}{Operating Systems, University Of Utah}

\section{Services}

\cvline{2022}{Artifact Evaluation Committee Member, OSDI'22}
\cvline{2021}{Artifact Evaluation Committee Member, SOSP'21}

\section{Bachelor Thesis}

\cvline{Title}{\emph{Power Efficient Remotely Operated Underwater Vehicle Using Buoyancy Chambers}}
\cvline{Supervisor}{Dr. Md. Mashud Karim}

\section{Skills}

\cvitem{Languages}{C, C++, Rust, CUDA, Python, Go, Assembly(ARM, X86)}
\cvitem{Tools}{CXL, eBPF, RDMA, DPDK, Pytorch, NVIDIA DOCA, LLVM, Linux perf, BOLT}
\cvitem{Platforms}{NVIDIA BlueField 2, AVR Microcontrollers, Raspberry Pi, Arduino}
\cvitem{Text Editing}{Vim, \LaTeX}

\end{document}
